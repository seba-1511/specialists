\author{Seb Arnold \break arnolds@usc.edu}
\title{Cluster Optimization for Specialist Networks}
\date{\today}
\documentclass[12pt]{article}

\input{\string~/tex_templates/core}

% Custom Title

%\def\maketitle{
    %\centering
    %\par\textbf{\LARGE\@title}
    %\par\hfill
    %\par{\@author, \@date}
    %\par\hfill
    %\par\hfill
    %\rule{\textwidth}{3pt}
%}

\def\maketitle{
    \begin{centering}
    \par\rule{\textwidth}{2pt}
    \par\hfill
    \par\textbf{\LARGE\@title}
    \par\hfill
    \par{\textit{\@author}}
    \par\hfill
    \par{\@date}
    \par\rule{\textwidth}{2pt}
    \end{centering}
}

\begin{document}
\thispagestyle{empty}
\maketitle
\hfill
\abstract
With the recent advances in deep neural networks, several experiments
involved the generalist-specialist paradigm for classification. However,
until now no formal study compared the performance of different
clustering algorithms for class assignment. In this paper we perform
such a study, suggest slight modifications to the clustering procedures,
and propose a novel algorithm designed to optimize the performance of of
the specialist-generalist classification system. Our experiments on the
CIFAR-10 and CIFAR-100 datasets allow us to investigate situations for
varying number of classes on similar data.

\section{Introduction}\label{introduction}

Designing an efficient classification system using deep neural networks
is a complicated task, which often use a multitude of models arranged in
ensembles. (Dieleman \& al. 2015, Simonyan \& Zisserman, 2015) Those
ensembles often lead to state-of-the-art results on a wide range of
different tasks such as image classification (Szegedy \& al. 2014),
speech recognition (Hannun \& al. 2014), and machine translation.
(Sutskever \& al. 2014) The models are trained independently and in
parallel, and different techniques can be used to merge their
predictions.

\begin{figure}[htbp]
\centering
\includegraphics{./figs/specialists.png}
\caption{An example of specialist architecture with three specialists}
\end{figure}

A more structured alternative to ensembling is the use of the
specialist-generalist framework. As described by Bochereau \& Bourgine
(1990), a natural analogy can be drawn from the medical field; a patient
first consults a general practitioner who provides an initial diagnosis
which is then refined by one or several specialists. In the case of
classification, the doctors are replaced by neural networks and the
final prediction is a combination of the specialists' outputs, and may
or may not include the generalist's take.

In recent years, generalist and specialists have been studied under
different circumstances. Hinton \& al. (2014) used specialists to create
an efficient image classifier for a large private dataset. The final
predictions of the specialists were then used to train a reduced
classifier that achieved performance similar to the whole ensemble.
Kahou \& al. (2015) describe a multimodal approach for emotion
recognition in videos, based on specialists. Maybe closer to our work,
Warde-Farley \& al. (2015) added ``auxiliary heads'' (acting as
specialists) to their baseline network, using the precomputed features
for both classification and clustering. They also underlined one of the
main advantages of using specialists; a relatively low (and
parallelizable) additional computational cost for increased performance.

\section{Clustering Algorithms}\label{clustering-algorithms}

In order to assign classes to the specialists networks, we compare
several clustering algorithms on the confusion matrix of the outputs of
the generalist. This confusion matrix is computed on a held-out
partition of the dataset. Following previous works, we started by
considering two baseline clustering algorithms, namely Lloyd's K-Means
algorithm and Spectral clustering, according to the formulation of Ng \&
al. (2002). In addition to those baseline algorithms, we evaluate the
performance of two novel procedures specifically designed to improve the
generalist-specialist paradigm. Those algorithms are described in the
following paragraphs, and pseudo code is given in the Appendix.

We also experimented with different ways of building a confusion matrix.
Besides the usual way (denoted here as \emph{standard}) we tried three
alternatives:

\begin{itemize}
\itemsep1pt\parskip0pt\parsep0pt
\item
  \emph{soft sum}: for each prediction, we use the raw model output
  instead of the one-hot multi-class output,
\item
  \emph{soft sum pred}: just like \emph{soft sum}, but only add the
  prediction output to the confusion matrix, if the class was correctly
  predicted,
\item
  \emph{soft sum not pred}: similarly to \emph{soft sum pred}, but only
  if the prediction output was incorrectly predicted.
\end{itemize}

As discussed in later sections, the influence of the confusion matrix is
minimal. Nonetheless we include them for completeness purposes.

Both of our clustering algorithms further modify the confusion matrix
$A$ by computing $CM = \textbf{A}^\top + \textbf{A}$, which symmetrizes
the matrix. We define the entries of the matrix to be the
\emph{animosity score} between two classes; given classes \emph{a} and
\emph{b}, their animosity score is found at $CM_{a, b}$. We then
initialize each cluster with non-overlapping pairs of classes yielding
maximal animosity score. We then greedily select the next classes to be
added to the clusters, according to the following rules:

\begin{itemize}
\item
  In the case of \emph{greedy single} clustering, a single class
  maximizing the overall animosity score is added to the cluster
  yielding the largest averaged sum of animosity towards this class.
  This partitions the classes in clusters, building on the intuition
  that classes that are hard to distinguish should be put together.
\item
  In the case of \emph{greedy pairs} clustering, we follow the same
  strategy as in \emph{greedy single} clustering but act on pair of
  classes instead of single classes. In this case we allow the clusters
  to share elements, and thus specialists can ahve overlapping
  jdugements.
\end{itemize}

This process is repeated until all classes have been assigned to at
least one cluster.

\section{Experiments}\label{experiments}

We investigate the performance of the aforementioned algorithms on the
CIFAR-10 and CIFAR-100 datasets (Krizhevsky, 2009). Both datasets
contain similar images, partitioned in 45'000 train, 5'000 validation,
and 10'000 test images. They contain 10 and 100 classes respectively.
For both experiments we train the generalist network on the train set
only, and use the validation set for clustering purposes. As we are
interested in the clustering performance we did not augment nor
pre-process the images. Note that when trained on the horizontally
flipped training and validation set our baseline algorithm reaches
10.18\% and 32.22\% misclassification error, which is competitive with
the current state-of-the-art presented in Springenberg \& al. (2015).

Following Courbariaux \& al (2015), the baseline network is based on the
conclusions of Simonyan \& al (2015) and uses three pairs of
batch-normalized convolutional layers, each followed by a max-pooling
layer, and two fully-connected layers. The same model is used for
specialists, whose weights are initialized with the trained weights of
the generalist. \footnote{The code for those experiments, is freely
  available online at
  \href{http://www.github.com/seba-1511/specialists}{github.com/seba-1511/specialists}.}
One major departure from the work of Hinton \& al. (2014) is that our
specialists are predicting over the same classes as the generalist,
i.e.~we do not merge all classes outside of the cluster into a unique
one. With regards to the generalist, a specialist is only biased towards
a subset of the classes, since it has been fine-tuned to perform well on
those ones.

\subsection{CIFAR-10}\label{cifar-10}

For CIFAR-10 experiments, we considered up to five clusters, and all of
the possible combinations of confusion matrix and clustering algorithms.
The results for this experiments are reported in Table 1.

\begin{longtable}[c]{@{}lllll@{}}
\toprule\addlinespace
Results & standard & soft sum & soft sum pred & soft sum not pred
\\\addlinespace
\midrule\endhead
spectral & (0.7342, 2) & (0.4117, 3) & (0.4541, 4) & (0.4143, 2)
\\\addlinespace
greedy singles & (0.2787, 3) & (0.2774, 2) & (0.3869, 4) & (0.2727, 2)
\\\addlinespace
kmeans & (0.8037, 2) & (0.8037, 2) & (0.8034, 2) & (0.804, 2)
\\\addlinespace
greedy pairs & (0.8584, 3) & (0.8483, 3) & (0.8473, 3) & (0.8611, 3)
\\\addlinespace
\bottomrule
\addlinespace
\caption{Experiment results for CIFAR-10}
\end{longtable}

Interestingly, the choice of confusion matrix has only a limited impact
on the overall performance, indicating that the emphasis should be put
on the the clustering algorithm. We notice that clustering with greedy
pairs consistantly yields better scores. However none of the specialist
experiments is able to improve on the baseline, indicating that
specialists might not be as efficient when dealing with a small number
of classes.

\subsection{CIFAR-100}\label{cifar-100}

For CIFAR-100 we performed the exact same experiment as for CIFAR-10 but
considered using more specialists, the largest experiments involving 28
clusters. The results are shown in Table 2.

\begin{longtable}[c]{@{}lllll@{}}
\toprule\addlinespace
Results & standard & soft sum & soft sum pred & soft sum not pred
\\\addlinespace
\midrule\endhead
spectral & (0.5828, 2) & (0.5713, 2) & (0.5755, 2) & (0.5795, 3)
\\\addlinespace
greedy singles & (0.3834, 2) & (0.3733, 2) & (0.3803, 2) & (0.3551, 2)
\\\addlinespace
kmeans & (0.5908, 2) & (0.5618, 2) & (0.5820, 3) & (0.5876, 2)
\\\addlinespace
greedy pairs & (0.6141, 6) & (0.5993, 6) & (0.6111, 6) & (0.607, 6)
\\\addlinespace
\bottomrule
\addlinespace
\caption{Experiment results for CIFAR-100}
\end{longtable}

Similarly to CIFAR-10, we observe that greedy pairs clustering
outperforms the other clustering techniques, and that the different
types of confusion matrix have a limited influence on the final score.
We also notice that fewer clusters tend to work better. Finally, and
unlike the results for CIFAR-10, some of the specialists are able to
improve upon the generalist, which confirms our intuition that
specialists are better suited to problems involving numerous output
classes.

Our explanation for the improved performance of greedy pairs is the
following. Allowing clusters to overlap leads to the assignment of
difficult classes to multiple specialists. At inference time, more
networks will influence the final prediction which is analogous to
building a larger ensemble for difficult classes.

\section{Conclusion and Future Work}\label{conclusion-and-future-work}

We introduced a novel clustering algorithm for the specialist-generalist
framework, which is able to consistantly outperform other techniques. We
also provided a preliminary study of the different factors coming into
play when dealing with specialists, and concluded that the choice of
confusion matrix from our proposed set only has little impact on the
final classification outcome.

Despite our encouraging results with clustering techniques, no one of
our specialists-based experiments came close to compete with the
generalist model trained on the entire train and validation set. This
was a surprising outcome and we suppose that this effect comes from the
size of the datasets. In both cases, 5'000 images corresponds to 10\% of
the original training set and removing that many train examples has a
drastic effect on both generalists and specialists. All the more so
since we are not using any kind of data augmentation techniques, which
could have moderated this downside. An obvious future step is to
validate the presented ideas on a much larger dataset such as Imagenet
(Russakovsky \& al. 2014) where splitting the train set would not hurt
the train score as much.

\subsubsection{Acknowledgments}\label{acknowledgments}

We would like to thank Greg Ver Steeg, Gabriel Pereyra, and Oriol
Vinyals for their comments and advices. We also thank Nervana Systems
for providing GPUs as well as their help with neon, their deep learning
framework.

\section{References}\label{references}

Bochereau, Laurent, and Bourgine, Paul. A Generalist-Specialist Paradigm
for Multilayer Neural Networks. Neural Networks, 1990.

Courbariaux, Matthieu, Bengio, Yoshua, and David, Jean-Pierre.
BinaryConnect: Training Deep Neural Networks with Binary Weights during
Propagations. NIPS, 2015.

Dieleman, Sander, Willett, Kyle W., and Dambre, Joni. Rotation-invarient
convolutional neural networks for galaxy morphology prediction. Oxford
Journals, 2015.

Hannun, Awni, Case, Carl, Casper, Jared, Catanzaro, Bryan, Diamos, Greg,
Elsen, Erich, Prenger, Ryan, Satheesh, Sanjeev, Sengupta, Shubho,
Coates, Adam, and Ng, Andrew Y. Deep Speach: Scaling up end-to-end
speech recognition. Arxiv Preprint, 2014.

Hinton, Geoffrey E., Vinyals, Oriol, and Dean, Jeff. Distilling th
Knowledge in a Neural Network. NIPS 2014 Deep Learning Workshop.

Kahou, Samira Ebrahimi, Bouthiller, Xavier, Lamblin, Pascal, Gulcehre,
Caglar, Michalski, Vincent, Konda, Kishore, Jean, Sébastien, Froumenty,
Pierre, Dauphin, Yann, Boulanger-Lewandowski, Nicolas, Ferrari, Raul
Chandias, Mirza, Mehdi, Warde-Farley, David, Courville, Aaron, Vincent,
Pascal, Memisevic, Roland, Pal, Christopher, and Bengio, Yoshua.
EmoNets: Multimodal deep learning approaches for emation recofnition in
video. Journal on Mutlimodal User Interfaces, 2015.

Krizhevsky, Alex. Learning Multiple Layers of Features from Tiny Images.
2009.

Ng, Andrew Y., Jordan, Micheal I., Weiss, Yair. On spectral clustering:
Analysis and an algorithm. NIPS 2002.

Russakovsky, Olga, Deng, Jia, Su, Hao, Krause, Jonathan, Satheesh,
Sanjeev, Ma, Sean, huang, Zhiheng, Karpathy, Andrej, Khosla, Aditya,
Bernstain, Michael, Berg, Alexander C., and Fei-Fei, Li. ImageNet Large
Scale Visual Recognition Challenge. International Journal of Computer
Vision, 2015.

Simonyan, Karen and Zisserman, Andrew. Very Deep Convolutional Networks
for Large-Scale Image Recognition. International Conference on Learning
Representations, 2015.

Springenberg, Jost Tobias, Dosovitskiy, Alexey, Brox, Thomas, and
Riedmiller, Martin. Striving for Simplicity: The All Convolutional Net.
International Conference on Learning Representations Workshop, 2015.

Sutskever, Ilya, Vinyals, Oriol, and Le, Quoc V. Sequence to Sequence
Learning with Neural Networks. Arxiv Preprint, 2014.

Szegedy, Christian, Liu, Wei, Jia, Yangqing, Sermanet, Pierre, Reed,
Scott, Anguelov, Dragomir, Erhan, Dumitru, Vanhoucke, Vincent, and
Rabinovich, Andrew. Going deeper with convolutions. Arxiv Preprint,
2014.

Warde-Farley, David, Rabinovich, Andrew, and Anguelov, Dragomir.
Self-Informed Neural Networks Structure Learning. International
Conference on Representations Learning, 2015.

\section{Appendix}\label{appendix}

\paragraph{Greedy Pairs Pseudo Code}\label{greedy-pairs-pseudo-code}

\begin{enumerate}
\def\labelenumi{\arabic{enumi}.}
\itemsep1pt\parskip0pt\parsep0pt
\item
  Given a confusion matrix M and N desired clusters.
\item
  $M \leftarrow M + M^T$
\item
  Initialize N clusters with non-overlapping pairs maximizing the
  entries of M.
\item
  Until each class has been assigned at least once:

  \begin{itemize}
  \itemsep1pt\parskip0pt\parsep0pt
  \item
    Get the next pair maximizing the entry in M
  \item
    Find which cluster minimizes the sum of animosity of both classes.
  \item
    Assign both classes to this cluster
  \end{itemize}
\item
  Return the clusters
\end{enumerate}

Note: A python implementation of both greedy pairs and greedy single can
be found at \url{http://www.github.com/seba-1511/specialists}.

\bibliographystyle{abbrv}
\bibliography{biblio}

%\end{multicols}
\end{document}
