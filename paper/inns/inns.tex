%%%%%%%%%%%%%%%%%%%% author.tex %%%%%%%%%%%%%%%%%%%%%%%%%%%%%%%%%%%
%
% sample root file for your "contribution" to a contributed volume
%
% Use this file as a template for your own input.
%
%%%%%%%%%%%%%%%% Springer %%%%%%%%%%%%%%%%%%%%%%%%%%%%%%%%%%


% RECOMMENDED %%%%%%%%%%%%%%%%%%%%%%%%%%%%%%%%%%%%%%%%%%%%%%%%%%%
\documentclass[graybox]{styles/svmult}

% choose options for [] as required from the list
% in the Reference Guide

\usepackage{mathptmx}       % selects Times Roman as basic font
\usepackage{helvet}         % selects Helvetica as sans-serif font
\usepackage{courier}        % selects Courier as typewriter font
\usepackage{type1cm}        % activate if the above 3 fonts are
                            % not available on your system
%
\usepackage{makeidx}         % allows index generation
\usepackage{graphicx}        % standard LaTeX graphics tool
                             % when including figure files
\usepackage{multicol}        % used for the two-column index
\usepackage[bottom]{footmisc}% places footnotes at page bottom

% CUSTOM SEB:
\usepackage{algorithm}
\usepackage{algorithmicx}
\usepackage[noend]{algpseudocode}

% see the list of further useful packages
% in the Reference Guide

\makeindex             % used for the subject index
                       % please use the style svind.ist with
                       % your makeindex program

%%%%%%%%%%%%%%%%%%%%%%%%%%%%%%%%%%%%%%%%%%%%%%%%%%%%%%%%%%%%%%%%%%%%%%%%%%%%%%%%%%%%%%%%%

\begin{document}

\title*{Greedily Assigning Classes to Neural Network Specialists}
% Use \titlerunning{Short Title} for an abbreviated version of
% your contribution title if the original one is too long
\author{Sebastien Arnold}
% Use \authorrunning{Short Title} for an abbreviated version of
% your contribution title if the original one is too long
\institute{Sebastien Arnold \at Department of Computer Science, University of Southern California, \email{arnolds@usc.edu}
}
%
% Use the package "url.sty" to avoid
% problems with special characters
% used in your e-mail or web address
%
\maketitle

\abstract*{
With the recent advances in deep neural networks, several experiments
involved the generalist-specialist paradigm for classification. However,
until now no formal study compared the performance of different
clustering algorithms for class assignment. In this paper we perform
such a study, suggest slight modifications to the clustering procedures,
and propose a novel algorithm designed to optimize the performance of of
the specialist-generalist classification system. Our experiments on the
CIFAR-10 and CIFAR-100 datasets allow us to investigate situations for
varying number of classes on similar data. We find that our
\emph{greedy\_pairs} clustering algorithm consistently outperforms other
alternatives, while the choice of the confusion matrix has little impact
on the final performance.
}

\abstract{
With the recent advances in deep neural networks, several experiments
involved the generalist-specialist paradigm for classification. However,
until now no formal study compared the performance of different
clustering algorithms for class assignment. In this paper we perform
such a study, suggest slight modifications to the clustering procedures,
and propose a novel algorithm designed to optimize the performance of of
the specialist-generalist classification system. Our experiments on the
CIFAR-10 and CIFAR-100 datasets allow us to investigate situations for
varying number of classes on similar data. We find that our
\emph{greedy\_pairs} clustering algorithm consistently outperforms other
alternatives, while the choice of the confusion matrix has little impact
on the final performance.
}


\section{Introduction}\label{introduction}

Designing an efficient classification system using deep neural networks
is a complicated task, which often use a multitude of models arranged in
ensembles. (\cite{galaxy}, \cite{vgg}) Those ensembles often lead to
state-of-the-art results on a wide range of different tasks such as
image classification (\cite{inception}), speech recognition
(\cite{deepspeech2}), and machine translation (\cite{seq2seq}). The
models are trained independently and in parallel, and different
techniques can be used to merge their predictions.

\begin{figure}[b]
\centering
\includegraphics[width=\textwidth]{./figs/specialists.png}
\caption{An example of specialist architecture with three specialists.}
\label{fig:1}
\end{figure}

A more structured alternative to ensembling is the use of the
specialist-generalist framework. As described by \cite{bochereau1990}, a
natural analogy can be drawn from the medical field; a patient first
consults a general practitioner who provides an initial diagnosis which
is then refined by one or several specialists. In the case of
classification, the doctors are replaced by neural networks and the
final prediction is a combination of the specialists' outputs, and may
or may not include the generalist's take.

In recent years, generalist and specialists have been studied under
different circumstances. \cite{darkknowledge} used specialists to create
an efficient image classifier for a large private dataset. The final
predictions of the specialists were then used to train a reduced
classifier that achieved performance similar to the whole ensemble.
\cite{emonets} describe a multimodal approach for emotion recognition in
videos, based on specialists. Maybe closer to our work,
\cite{wardefarley} added ``auxiliary heads'' (acting as specialists) to
their baseline network, using the precomputed features for both
classification and clustering. They also underlined one of the main
advantages of using specialists; a relatively low (and parallelizable)
additional computational cost for increased performance.

\section{Clustering Algorithms}\label{clustering-algorithms}

In order to assign classes to the specialist networks, we compare
several clustering algorithms on the confusion matrix of the outputs of
the generalist. This confusion matrix is computed on a held-out
partition of the dataset. Following previous works, we started by
considering two baseline clustering algorithms, namely Lloyd's K-Means
algorithm and Spectral clustering, according to the formulation of
\cite{spectral}. In addition to those baseline algorithms, we evaluate
the performance of two novel procedures specifically designed to improve
the generalist-specialist paradigm. Those algorithms are described in
the following paragraphs, and pseudo code is given in the Appendix.

We also experimented with different ways of building the confusion
matrix. Besides the usual way (denoted here as \emph{standard}) we tried
three alternatives:

\begin{itemize}
\itemsep1pt\parskip0pt\parsep0pt
\item
  \emph{softsum}: for each prediction, we use the raw model output
  instead of the one-hot multi-class output,
\item
  \emph{softsum pred}: just like \emph{softsum}, but only add the
  prediction output to the confusion matrix, if the class was correctly
  predicted,
\item
  \emph{softsum not pred}: like to \emph{softsum pred}, but only if
  the prediction output was incorrectly predicted.
\end{itemize}

As discussed in later sections, the influence of the confusion matrix is
minimal. Nonetheless we include them for completeness purposes.

Both of our clustering algorithms further modify the confusion matrix
$A$ by computing $CM = \textbf{A}^\top + \textbf{A}$, which symmetrizes
the matrix. We define the entries of the matrix to be the
\emph{animosity score} between two classes; given classes \emph{a} and
\emph{b}, their animosity score is found at $CM_{a, b}$. We then
initialize each cluster with non-overlapping pairs of classes yielding
maximal animosity score. Finally, we greedily select the next classes to
be added to the clusters, according to the following rules:

\begin{itemize}
\item
  In the case of \emph{greedy single} clustering, a single class
  maximizing the overall animosity score is added to the cluster
  yielding the largest averaged sum of animosity towards this class.
  This partitions the classes in clusters, building on the intuition
  that classes that are hard to distinguish should be put together.
\item
  In the case of \emph{greedy pairs} clustering, we follow the same
  strategy as in \emph{greedy single} clustering but act on pair of
  classes instead of single classes. In this case we allow the clusters
  to overlap, and one prediction might include the opinion of several
  specialists.
\end{itemize}

This process is repeated until all classes have been assigned to at
least one cluster.

\section{Experiments}\label{experiments}

We investigate the performance of the aforementioned algorithms on the
CIFAR-10 and CIFAR-100 datasets (\cite{cifar}). Both datasets contain
similar images, partitioned in 45'000 train, 5'000 validation, and
10'000 test images. They contain 10 and 100 classes respectively. For
both experiments we train the generalist network on the train set only,
and use the validation set for clustering purposes. As we are interested
in the clustering performance we did not augment nor pre-process the
images. Note that when trained on the horizontally flipped training and
validation set our baseline algorithm reaches 10.18\% and 32.22\%
misclassification error respectively, which is competitive with the
current state-of-the-art presented in \cite{allcnn}.

Following \cite{binaryconnect}, the baseline network is based on the
conclusions of \cite{vgg} and uses three pairs of batch-normalized
convolutional layers, each followed by a max-pooling layer, and two
fully-connected layers. The same model is used for specialists, whose
weights are initialized with the trained weights of the generalist.
\footnote{The code for these experiments, is freely available online at
  \href{http://www.github.com/seba-1511/specialists}.}
One major departure from the work of \cite{darkknowledge} is that our
specialists are predicting over the same classes as the generalist,
i.e.~we do not merge all classes outside of the cluster into a unique
one. With regards to the generalist, a specialist is only biased towards
a subset of the classes, since it has been fine-tuned to perform well on
those ones.

\subsection{CIFAR-10}\label{cifar-10}

For CIFAR-10 experiments, we considered up to five clusters, and all of
the possible combinations of confusion matrix and clustering algorithms.
The results for this experiments are reported in Table 1.

\begin{table}
\caption{Experiment results for CIFAR-10}
\label{tab:1}       % Give a unique label
\begin{tabular}{p{3.2cm}p{2.0cm}p{2.0cm}p{2.0cm}p{2.0cm}}
\hline\noalign{\smallskip}
Clustering Algorithm & standard & softsum & softsum pred & softsum not pred \\
\noalign{\smallskip}\svhline\noalign{\smallskip}
spectral & (0.7046, 2) & (0.7719, 2) & (0.6989, 2) & (0.706, 2) \\
greedy singles & (0.5873, 2) & (0.5049, 2) & (0.5139, 3) & (0.5873, 2) \\
kmeans & (0.8202, 2) & (0.8202, 2) & (0.8202, 2) & (0.8202, 2) \\
greedy pairs & (0.8835, 2) & (0.8835, 2) & (0.8727, 3) & (0.8835, 2) \\
\noalign{\smallskip}\hline\noalign{\smallskip}
\end{tabular}
\end{table}

Interestingly, the choice of confusion matrix has only a limited impact
on the overall performance, indicating that the emphasis should be put
on the clustering algorithm. We notice that clustering with greedy pairs
consistently yields better scores. However none of the specialist
experiments is able to improve on the baseline, suggesting that
specialists might not be the framework of choice when dealing with a
small number of classes.

\subsection{CIFAR-100}\label{cifar-100}

For CIFAR-100 we performed the exact same experiment as for CIFAR-10 but
used more specialists, the largest experiments involving 28 clusters.
The results are shown in Table 2.

\begin{table}
\caption{Experiment results for CIFAR-100}
\label{tab:2}       % Give a unique label
\begin{tabular}{p{3.2cm}p{2.0cm}p{2.0cm}p{2.0cm}p{2.0cm}}
\hline\noalign{\smallskip}
Clustering Algorithm & standard & softsum & softsum pred & softsum not pred \\
\noalign{\smallskip}\svhline\noalign{\smallskip}
spectral & (0.5828, 2) & (0.5713, 2) & (0.5755, 2) & (0.5795, 3) \\
greedy singles & (0.3834, 2) & (0.3733, 2) & (0.3803, 2) & (0.3551, 2) \\
kmeans & (0.5908, 2) & (0.5618, 2) & (0.5820, 3) & (0.5876, 2) \\
greedy pairs & (0.6141, 6) & (0.5993, 6) & (0.6111, 6) & (0.607, 6) \\
\noalign{\smallskip}\hline\noalign{\smallskip}
\end{tabular}
\end{table}

Similarly to CIFAR-10, we observe that greedy pairs clustering
outperforms the other clustering techniques, and that the different
types of confusion matrix have a limited influence on the final score.
We also notice that fewer clusters tend to work better. Finally, and
unlike the results for CIFAR-10, some of the specialists are able to
improve upon the generalist, which confirms our intuition that
specialists are better suited to problems involving numerous output
classes.

We suggest the following explanation for the improved performance of
greedy pairs is the following. Allowing clusters to overlap leads to the
assignment of difficult classes to multiple specialists. At inference
time, more networks will influence the final prediction which is
analogous to building a larger ensemble for difficult classes.

\section{Conclusion and Future Work}\label{conclusion-and-future-work}

We introduced a novel clustering algorithm for the specialist-generalist
framework, which is able to consistently outperform other techniques. We
also provided a preliminary study of the different factors coming into
play when dealing with specialists, and concluded that the choice of
confusion matrix from our proposed set only has little impact on the
final classification outcome.

Despite our encouraging results with clustering techniques, no one of
our specialists-based experiments came close to compete with the
generalist model trained on the entire train and validation set. This
was a surprising outcome and we suppose that this effect comes from the
size of the datasets. In both cases, 5'000 images corresponds to 10\% of
the original training set and removing that many training examples has a
drastic effect on both generalists and specialists. All the more so
since we are not using any kind of data augmentation techniques, which
could have moderated this downside. An obvious future step is to
validate the presented ideas on a much larger dataset such as
\cite{imagenet} where splitting the train set would not hurt the train
score as much.

\acknowledgement

We would like to thank Greg Ver Steeg, Gabriel Pereyra, and Pranav Rajpurkar for their comments and advices. We also thank Nervana Systems
for providing GPUs as well as their help with their deep learning
framework.

\section{Appendix}\label{appendix}

%\subsection{Greedy Pairs Pseudo Code}\label{greedy-pairs-pseudo-code}

\begin{algorithm}
    \caption{Greedy Pairs Clustering}
    \label{greedy_pairs}
    \begin{algorithmic}[1] % The number tells where the line numbering should start
        \Procedure{GreedyPairs}{$M,N$} \Comment{Confusion matrix M, number of clusters N}
            \State $M\gets M + M^T$
            \State Initialize N clusters with non-overlapping pairs maximizing the entries of M.
            \While{every class has not been assigned}
                \State Get the next pair $(a, b)$ maximizing the entry in M
                \State cluster = $\underset{\text{c in clusters}}{\mathrm{argmin}}$(Animosity(a, c) + Animosity(b, c))
                \State Assign(cluster, a, b)
            \EndWhile\label{euclidendwhile}
            \State \textbf{return} clusters
        \EndProcedure
    \end{algorithmic}
\end{algorithm}

%Note: A python implementation of both greedy pairs and greedy single can
%be found at \url{http://www.github.com/seba-1511/specialists}.

%%%%%%%%%%%%%%%%%%%%%%%% referenc.tex %%%%%%%%%%%%%%%%%%%%%%%%%%%%%%
% sample references
% %
% Use this file as a template for your own input.
%
%%%%%%%%%%%%%%%%%%%%%%%% Springer-Verlag %%%%%%%%%%%%%%%%%%%%%%%%%%
%
% BibTeX users please use
 %\bibliographystyle{}
 %\bibliography{}
%
%\biblstarthook{References may be \textit{cited} in the text either by number (preferred) or by author/year.\footnote{Make sure that all references from the list are cited in the text. Those not cited should be moved to a separate \textit{Further Reading} section or chapter.} The reference list should ideally be \textit{sorted} in alphabetical order -- even if reference numbers are used for the their citation in the text. If there are several works by the same author, the following order should be used: 
%\begin{enumerate}
%\item all works by the author alone, ordered chronologically by year of publication
%\item all works by the author with a coauthor, ordered alphabetically by coauthor
%\item all works by the author with several coauthors, ordered chronologically by year of publication.
%\end{enumerate}
%The \textit{styling} of references\footnote{Always use the standard abbreviation of a journal's name according to the ISSN \textit{List of Title Word Abbreviations}, see \url{http://www.issn.org/en/node/344}} depends on the subject of your book:
%\begin{itemize}
%\item The \textit{two} recommended styles for references in books on \textit{mathematical, physical, statistical and computer sciences} are depicted in ~\cite{science-contrib, science-online, science-mono, science-journal, science-DOI} and ~\cite{phys-online, phys-mono, phys-journal, phys-DOI, phys-contrib}.
%\item Examples of the most commonly used reference style in books on \textit{Psychology, Social Sciences} are~\cite{psysoc-mono, psysoc-online,psysoc-journal, psysoc-contrib, psysoc-DOI}.
%\item Examples for references in books on \textit{Humanities, Linguistics, Philosophy} are~\cite{humlinphil-journal, humlinphil-contrib, humlinphil-mono, humlinphil-online, humlinphil-DOI}.
%\item Examples of the basic Springer style used in publications on a wide range of subjects such as \textit{Computer Science, Economics, Engineering, Geosciences, Life Sciences, Medicine, Biomedicine} are ~\cite{basic-contrib, basic-online, basic-journal, basic-DOI, basic-mono}. 
%\end{itemize}
%}

\begin{thebibliography}{99.}%


    \bibitem{bochereau1990}Bochereau, Laurent, and Bourgine, Paul. A Generalist-Specialist Paradigm for
Multilayer Neural Networks. Neural Networks, 1990.


\bibitem{binaryconnect}Courbariaux, Matthieu, Bengio, Yoshua, and David, Jean-Pierre. BinaryConnect:
Training Deep Neural Networks with Binary Weights during Propagations. NIPS,
2015.


\bibitem{galaxy}Dieleman, Sander, Willett, Kyle W., and Dambre, Joni. Rotation-invarient
convolutional neural networks for galaxy morphology prediction. Oxford Journals,
2015.


\bibitem{deepspeech2}Hannun, Awni, Case, Carl, Casper, Jared, Catanzaro, Bryan, Diamos, Greg, Elsen,
Erich, Prenger, Ryan, Satheesh, Sanjeev, Sengupta, Shubho, Coates, Adam, and Ng,
Andrew Y. Deep Speach: Scaling up end-to-end speech recognition. Arxiv Preprint,
2014.


\bibitem{darkknowledge}Hinton, Geoffrey E., Vinyals, Oriol, and Dean, Jeff. Distilling th Knowledge in
a Neural Network. NIPS 2014 Deep Learning Workshop.


\bibitem{emonets}Kahou, Samira Ebrahimi, Bouthiller, Xavier, Lamblin, Pascal, Gulcehre, Caglar,
Michalski, Vincent, Konda, Kishore, Jean, S�bastien, Froumenty, Pierre, Dauphin,
Yann, Boulanger-Lewandowski, Nicolas, Ferrari, Raul Chandias, Mirza, Mehdi,
Warde-Farley, David, Courville, Aaron, Vincent, Pascal, Memisevic, Roland, Pal,
Christopher, and Bengio, Yoshua. EmoNets: Multimodal deep learning approaches
for emation recofnition in video. Journal on Mutlimodal User Interfaces, 2015.


\bibitem{cifar}Krizhevsky, Alex. Learning Multiple Layers of Features from Tiny Images. 2009.


\bibitem{spectral}Ng, Andrew Y., Jordan, Micheal I., Weiss, Yair. On spectral clustering: Analysis
and an algorithm. NIPS 2002.


\bibitem{imagenet}Russakovsky, Olga, Deng, Jia, Su, Hao, Krause, Jonathan, Satheesh, Sanjeev, Ma,
Sean, huang, Zhiheng, Karpathy, Andrej, Khosla, Aditya, Bernstain, Michael,
Berg, Alexander C., and Fei-Fei, Li. ImageNet Large Scale Visual Recognition
Challenge. International Journal of Computer Vision, 2015.


\bibitem{vgg}Simonyan, Karen and Zisserman, Andrew. Very Deep Convolutional Networks for
Large-Scale Image Recognition. International Conference on Learning
Representations, 2015.


\bibitem{allcnn}Springenberg, Jost Tobias, Dosovitskiy, Alexey, Brox, Thomas, and Riedmiller,
Martin. Striving for Simplicity: The All Convolutional Net. International
Conference on Learning Representations Workshop, 2015.


\bibitem{seq2seq}Sutskever, Ilya, Vinyals, Oriol, and Le, Quoc V. Sequence to Sequence Learning with
Neural Networks. Arxiv Preprint, 2014.


\bibitem{inception}Szegedy, Christian, Liu, Wei, Jia, Yangqing, Sermanet, Pierre, Reed, Scott,
Anguelov, Dragomir, Erhan, Dumitru, Vanhoucke, Vincent, and Rabinovich, Andrew.
Going deeper with convolutions. Arxiv Preprint, 2014.


\bibitem{wardefarley}Warde-Farley, David, Rabinovich, Andrew, and  Anguelov, Dragomir. Self-Informed
Neural Networks Structure Learning. International Conference on Representations
Learning, 2015.

%@inproceedings{bochereau1990,
  %title={A generalist-specialist paradigm for multilayer neural networks},
  %author={Bochereau, Laurent and Bourgine, Paul},
  %booktitle={Neural Networks, 1990., 1990 IJCNN International Joint Conference on},
  %pages={87--91},
  %year={1990},
  %organization={IEEE}
%}

%@inproceedings{binaryconnect,
  %title={BinaryConnect: Training Deep Neural Networks with binary weights during propagations},
  %author={Courbariaux, Matthieu and Bengio, Yoshua and David, Jean-Pierre},
  %booktitle={Advances in Neural Information Processing Systems},
  %pages={3105--3113},
  %year={2015}
%}

%@article{galaxy,
  %title={Rotation-invariant convolutional neural networks for galaxy morphology prediction},
  %author={Dieleman, Sander and Willett, Kyle W and Dambre, Joni},
  %journal={Monthly Notices of the Royal Astronomical Society},
  %volume={450},
  %number={2},
  %pages={1441--1459},
  %year={2015},
  %publisher={Oxford University Press}
%}

%@article{deepspeech2,
  %title={Deep Speech 2: End-to-End Speech Recognition in English and Mandarin},
  %author={Amodei, Dario and Anubhai, Rishita and Battenberg, Eric and Case, Carl and Casper, Jared and Catanzaro, Bryan and Chen, Jingdong and Chrzanowski, Mike and Coates, Adam and Diamos, Greg and others},
  %journal={arXiv preprint arXiv:1512.02595},
  %year={2015}
%}

%@article{darkknowledge,
  %title={Distilling the knowledge in a neural network},
  %author={Hinton, Geoffrey and Vinyals, Oriol and Dean, Jeff},
  %journal={arXiv preprint arXiv:1503.02531},
  %year={2015}
%}

%@article{emonets,
  %title={Emonets: Multimodal deep learning approaches for emotion recognition in video},
  %author={Kahou, Samira Ebrahimi and Bouthillier, Xavier and Lamblin, Pascal and Gulcehre, Caglar and Michalski, Vincent and Konda, Kishore and Jean, S{\'e}bastien and Froumenty, Pierre and Dauphin, Yann and Boulanger-Lewandowski, Nicolas and others},
  %journal={Journal on Multimodal User Interfaces},
  %pages={1--13},
  %publisher={Springer}
%}

%@misc{cifar,
  %title={Learning multiple layers of features from tiny images},
  %author={Krizhevsky, Alex},
  %year={2009},
  %publisher={Citeseer}
%}

%@article{spectral,
  %title={On Spectral Clustering: Analysis and an algorithm},
  %author={Ng, Andrew Y and Jordan, Michael I and Weiss, Yair}
%}

%@article{imagenet,
  %title={Imagenet large scale visual recognition challenge},
  %author={Russakovsky, Olga and Deng, Jia and Su, Hao and Krause, Jonathan and Satheesh, Sanjeev and Ma, Sean and Huang, Zhiheng and Karpathy, Andrej and Khosla, Aditya and Bernstein, Michael and others},
  %journal={International Journal of Computer Vision},
  %volume={115},
  %number={3},
  %pages={211--252},
  %year={2015},
  %publisher={Springer}
%}

%@article{vgg,
  %title={Very deep convolutional networks for large-scale image recognition},
  %author={Simonyan, Karen and Zisserman, Andrew},
  %journal={arXiv preprint arXiv:1409.1556},
  %year={2014}
%}

%@article{allcnn,
  %title={Striving for simplicity: The all convolutional net},
  %author={Springenberg, Jost Tobias and Dosovitskiy, Alexey and Brox, Thomas and Riedmiller, Martin},
  %journal={arXiv preprint arXiv:1412.6806},
  %year={2014}
%}

%@inproceedings{seq2seq,
  %title={Sequence to sequence learning with neural networks},
  %author={Sutskever, Ilya and Vinyals, Oriol and Le, Quoc V},
  %booktitle={Advances in neural information processing systems},
  %pages={3104--3112},
  %year={2014}
%}

%@inproceedings{inception,
  %title={Going deeper with convolutions},
  %author={Szegedy, Christian and Liu, Wei and Jia, Yangqing and Sermanet, Pierre and Reed, Scott and Anguelov, Dragomir and Erhan, Dumitru and Vanhoucke, Vincent and Rabinovich, Andrew},
  %booktitle={Proceedings of the IEEE Conference on Computer Vision and Pattern Recognition},
  %pages={1--9},
  %year={2015}
%}

%@article{wardefarley,
  %title={Self-informed neural network structure learning},
  %author={Warde-Farley, David and Rabinovich, Andrew and Anguelov, Dragomir},
  %journal={arXiv preprint arXiv:1412.6563},
  %year={2014}
%}




% and use \bibitem to create references.
%
% Use the following syntax and markup for your references if 
% the subject of your book is from the field 
% "Mathematics, Physics, Statistics, Computer Science"
%
% Contribution 
%\bibitem{science-contrib} Broy, M.: Software engineering --- from auxiliary to key technologies. In: Broy, M., Dener, E. (eds.) Software Pioneers, pp. 10-13. Springer, Heidelberg (2002)
%%
%% Online Document
%\bibitem{science-online} Dod, J.: Effective substances. In: The Dictionary of Substances and Their Effects. Royal Society of Chemistry (1999) Available via DIALOG. \\
%\url{http://www.rsc.org/dose/title of subordinate document. Cited 15 Jan 1999}
%%
%% Monograph
%\bibitem{science-mono} Geddes, K.O., Czapor, S.R., Labahn, G.: Algorithms for Computer Algebra. Kluwer, Boston (1992) 
%%
%% Journal article
%\bibitem{science-journal} Hamburger, C.: Quasimonotonicity, regularity and duality for nonlinear systems of partial differential equations. Ann. Mat. Pura. Appl. \textbf{169}, 321--354 (1995)
%%
%% Journal article by DOI
%\bibitem{science-DOI} Slifka, M.K., Whitton, J.L.: Clinical implications of dysregulated cytokine production. J. Mol. Med. (2000) doi: 10.1007/s001090000086 
%%
%\bigskip

% Use the following (APS) syntax and markup for your references if 
% the subject of your book is from the field 
% "Mathematics, Physics, Statistics, Computer Science"
%
% Online Document
%\bibitem{phys-online} J. Dod, in \textit{The Dictionary of Substances and Their Effects}, Royal Society of Chemistry. (Available via DIALOG, 1999), 
%\url{http://www.rsc.org/dose/title of subordinate document. Cited 15 Jan 1999}
%%
%% Monograph
%\bibitem{phys-mono} H. Ibach, H. L\"uth, \textit{Solid-State Physics}, 2nd edn. (Springer, New York, 1996), pp. 45-56 
%%
%% Journal article
%\bibitem{phys-journal} S. Preuss, A. Demchuk Jr., M. Stuke, Appl. Phys. A \textbf{61}
%%
%% Journal article by DOI
%\bibitem{phys-DOI} M.K. Slifka, J.L. Whitton, J. Mol. Med., doi: 10.1007/s001090000086
%%
%% Contribution 
%\bibitem{phys-contrib} S.E. Smith, in \textit{Neuromuscular Junction}, ed. by E. Zaimis. Handbook of Experimental Pharmacology, vol 42 (Springer, Heidelberg, 1976), p. 593
%%
%\bigskip
%%
%% Use the following syntax and markup for your references if 
%% the subject of your book is from the field 
%% "Psychology, Social Sciences"
%%
%%
%% Monograph
%\bibitem{psysoc-mono} Calfee, R.~C., \& Valencia, R.~R. (1991). \textit{APA guide to preparing manuscripts for journal publication.} Washington, DC: American Psychological Association.
%%
%% Online Document
%\bibitem{psysoc-online} Dod, J. (1999). Effective substances. In: The dictionary of substances and their effects. Royal Society of Chemistry. Available via DIALOG. \\
%\url{http://www.rsc.org/dose/Effective substances.} Cited 15 Jan 1999.
%%
%% Journal article
%\bibitem{psysoc-journal} Harris, M., Karper, E., Stacks, G., Hoffman, D., DeNiro, R., Cruz, P., et al. (2001). Writing labs and the Hollywood connection. \textit{J Film} Writing, 44(3), 213--245.
%%
%% Contribution 
%\bibitem{psysoc-contrib} O'Neil, J.~M., \& Egan, J. (1992). Men's and women's gender role journeys: Metaphor for healing, transition, and transformation. In B.~R. Wainrig (Ed.), \textit{Gender issues across the life cycle} (pp. 107--123). New York: Springer.
%%
%% Journal article by DOI
%\bibitem{psysoc-DOI}Kreger, M., Brindis, C.D., Manuel, D.M., Sassoubre, L. (2007). Lessons learned in systems change initiatives: benchmarks and indicators. \textit{American Journal of Community Psychology}, doi: 10.1007/s10464-007-9108-14.
%%
%%
%% Use the following syntax and markup for your references if 
%% the subject of your book is from the field 
%% "Humanities, Linguistics, Philosophy"
%%
%\bigskip
%%
%% Journal article
%\bibitem{humlinphil-journal} Alber John, Daniel C. O'Connell, and Sabine Kowal. 2002. Personal perspective in TV interviews. \textit{Pragmatics} 12:257--271
%%
%% Contribution 
%\bibitem{humlinphil-contrib} Cameron, Deborah. 1997. Theoretical debates in feminist linguistics: Questions of sex and gender. In \textit{Gender and discourse}, ed. Ruth Wodak, 99--119. London: Sage Publications.
%%
%% Monograph
%\bibitem{humlinphil-mono} Cameron, Deborah. 1985. \textit{Feminism and linguistic theory.} New York: St. Martin's Press.
%%
%% Online Document
%\bibitem{humlinphil-online} Dod, Jake. 1999. Effective substances. In: The dictionary of substances and their effects. Royal Society of Chemistry. Available via DIALOG. \\
%http://www.rsc.org/dose/title of subordinate document. Cited 15 Jan 1999
%%
%% Journal article by DOI
%\bibitem{humlinphil-DOI} Suleiman, Camelia, Daniel C. O�Connell, and Sabine Kowal. 2002. `If you and I, if we, in this later day, lose that sacred fire...�': Perspective in political interviews. \textit{Journal of Psycholinguistic Research}. doi: 10.1023/A:1015592129296.
%%
%%
%%
%\bigskip
%%
%%
%% Use the following syntax and markup for your references if 
%% the subject of your book is from the field 
%% "Computer Science, Economics, Engineering, Geosciences, Life Sciences"
%%
%%
%% Contribution 
%\bibitem{basic-contrib} Brown B, Aaron M (2001) The politics of nature. In: Smith J (ed) The rise of modern genomics, 3rd edn. Wiley, New York 
%%
%% Online Document
%\bibitem{basic-online} Dod J (1999) Effective Substances. In: The dictionary of substances and their effects. Royal Society of Chemistry. Available via DIALOG. \\
%\url{http://www.rsc.org/dose/title of subordinate document. Cited 15 Jan 1999}
%%
%% Journal article by DOI
%\bibitem{basic-DOI} Slifka MK, Whitton JL (2000) Clinical implications of dysregulated cytokine production. J Mol Med, doi: 10.1007/s001090000086
%%
%% Journal article
%\bibitem{basic-journal} Smith J, Jones M Jr, Houghton L et al (1999) Future of health insurance. N Engl J Med 965:325--329
%%
%% Monograph
%\bibitem{basic-mono} South J, Blass B (2001) The future of modern genomics. Blackwell, London 
%
\end{thebibliography}


%\begin{thebibliography}
%%\bibliographystyle{apalike}
%\bibliographystyle{}

%\bibliography{biblio}

%\end{thebibliography}

%\end{multicols}
\end{document}

%\section{Section Heading}
%\label{sec:1}
%Use the template \emph{chapter.tex} together with the Springer document class SVMono (monograph-type books) or SVMult (edited books) to style the various elements of your chapter content in the Springer layout.

%Instead of simply listing headings of different levels we recommend to
%let every heading be followed by at least a short passage of text.
%Further on please use the \LaTeX\ automatism for all your
%cross-references and citations. And please note that the first line of
%text that follows a heading is not indented, whereas the first lines of
%all subsequent paragraphs are.

%\section{Section Heading}
%\label{sec:2}
%% Always give a unique label
%% and use \ref{<label>} for cross-references
%% and \cite{<label>} for bibliographic references
%% use \sectionmark{}
%% to alter or adjust the section heading in the running head
%Instead of simply listing headings of different levels we recommend to
%let every heading be followed by at least a short passage of text.
%Further on please use the \LaTeX\ automatism for all your
%cross-references and citations.

%Please note that the first line of text that follows a heading is not indented, whereas the first lines of all subsequent paragraphs are.

%Use the standard \verb|equation| environment to typeset your equations, e.g.
%%
%\begin{equation}
%a \times b = c\;,
%\end{equation}
%%
%however, for multiline equations we recommend to use the \verb|eqnarray| environment\footnote{In physics texts please activate the class option \texttt{vecphys} to depict your vectors in \textbf{\itshape boldface-italic} type - as is customary for a wide range of physical subjects}.
%\begin{eqnarray}
%a \times b = c \nonumber\\
%\vec{a} \cdot \vec{b}=\vec{c}
%\label{eq:01}
%\end{eqnarray}

%\subsection{Subsection Heading}
%\label{subsec:2}
%Instead of simply listing headings of different levels we recommend to
%let every heading be followed by at least a short passage of text.
%Further on please use the \LaTeX\ automatism for all your
%cross-references\index{cross-references} and citations\index{citations}
%as has already been described in Sect.~\ref{sec:2}.

%\begin{quotation}
%Please do not use quotation marks when quoting texts! Simply use the \verb|quotation| environment -- it will automatically render Springer's preferred layout.
%\end{quotation}


%\subsubsection{Subsubsection Heading}
%Instead of simply listing headings of different levels we recommend to
%let every heading be followed by at least a short passage of text.
%Further on please use the \LaTeX\ automatism for all your
%cross-references and citations as has already been described in
%Sect.~\ref{subsec:2}, see also Fig.~\ref{fig:1}\footnote{If you copy
%text passages, figures, or tables from other works, you must obtain
%\textit{permission} from the copyright holder (usually the original
%publisher). Please enclose the signed permission with the manuscript. The
%sources\index{permission to print} must be acknowledged either in the
%captions, as footnotes or in a separate section of the book.}

%Please note that the first line of text that follows a heading is not indented, whereas the first lines of all subsequent paragraphs are.

%% For figures use
%%
%%\begin{figure}[b]
%%\sidecaption
%%% Use the relevant command for your figure-insertion program
%%% to insert the figure file.
%%% For example, with the graphicx style use
%%\includegraphics[scale=.65]{figure}
%%%
%%% If no graphics program available, insert a blank space i.e. use
%%%\picplace{5cm}{2cm} % Give the correct figure height and width in cm
%%%
%%\caption{If the width of the figure is less than 7.8 cm use the \texttt{sidecapion} command to flush the caption on the left side of the page. If the figure is positioned at the top of the page, align the sidecaption with the top of the figure -- to achieve this you simply need to use the optional argument \texttt{[t]} with the \texttt{sidecaption} command}
%%\label{fig:1}       % Give a unique label
%%\end{figure}

%\begin{figure}[htbp]
%\label{fig:1}       % Give a unique label
%\centering
%\includegraphics{./figs/specialists.png}
%\caption{An example of specialist architecture with three specialists}
%\end{figure}

%\paragraph{Paragraph Heading} %
%Instead of simply listing headings of different levels we recommend to
%let every heading be followed by at least a short passage of text.
%Further on please use the \LaTeX\ automatism for all your
%cross-references and citations as has already been described in
%Sect.~\ref{sec:2}.

%Please note that the first line of text that follows a heading is not indented, whereas the first lines of all subsequent paragraphs are.

%For typesetting numbered lists we recommend to use the \verb|enumerate| environment -- it will automatically render Springer's preferred layout.

%\begin{enumerate}
%\item{Livelihood and survival mobility are oftentimes coutcomes of uneven socioeconomic development.}
%\begin{enumerate}
%\item{Livelihood and survival mobility are oftentimes coutcomes of uneven socioeconomic development.}
%\item{Livelihood and survival mobility are oftentimes coutcomes of uneven socioeconomic development.}
%\end{enumerate}
%\item{Livelihood and survival mobility are oftentimes coutcomes of uneven socioeconomic development.}
%\end{enumerate}


%\subparagraph{Subparagraph Heading} In order to avoid simply listing headings of different levels we recommend to let every heading be followed by at least a short passage of text. Use the \LaTeX\ automatism for all your cross-references and citations as has already been described in Sect.~\ref{sec:2}, see also Fig.~\ref{fig:2}.

%For unnumbered list we recommend to use the \verb|itemize| environment -- it will automatically render Springer's preferred layout.

%\begin{itemize}
%\item{Livelihood and survival mobility are oftentimes coutcomes of uneven socioeconomic development, cf. Table~\ref{tab:1}.}
%\begin{itemize}
%\item{Livelihood and survival mobility are oftentimes coutcomes of uneven socioeconomic development.}
%\item{Livelihood and survival mobility are oftentimes coutcomes of uneven socioeconomic development.}
%\end{itemize}
%\item{Livelihood and survival mobility are oftentimes coutcomes of uneven socioeconomic development.}
%\end{itemize}

%%\begin{figure}[t]
%%\sidecaption[t]
%%% Use the relevant command for your figure-insertion program
%%% to insert the figure file.
%%% For example, with the option graphics use
%%\includegraphics[scale=.65]{figure}
%%%
%%% If no graphics program available, insert a blank space i.e. use
%%%\picplace{5cm}{2cm} % Give the correct figure height and width in cm
%%%
%%%\caption{Please write your figure caption here}
%%\caption{If the width of the figure is less than 7.8 cm use the \texttt{sidecapion} command to flush the caption on the left side of the page. If the figure is positioned at the top of the page, align the sidecaption with the top of the figure -- to achieve this you simply need to use the optional argument \texttt{[t]} with the \texttt{sidecaption} command}
%%\label{fig:2}       % Give a unique label
%%\end{figure}

%\runinhead{Run-in Heading Boldface Version} Use the \LaTeX\ automatism for all your cross-references and citations as has already been described in Sect.~\ref{sec:2}.

%\subruninhead{Run-in Heading Italic Version} Use the \LaTeX\ automatism for all your cross-refer\-ences and citations as has already been described in Sect.~\ref{sec:2}\index{paragraph}.
%% Use the \index{} command to code your index words
%%
%% For tables use
%%
%\begin{table}
%\caption{Please write your table caption here}
%\label{tab:1}       % Give a unique label
%%
%% Follow this input for your own table layout
%%
%\begin{tabular}{p{2cm}p{2.4cm}p{2cm}p{4.9cm}}
%\hline\noalign{\smallskip}
%Classes & Subclass & Length & Action Mechanism  \\
%\noalign{\smallskip}\svhline\noalign{\smallskip}
%Translation & mRNA$^a$  & 22 (19--25) & Translation repression, mRNA cleavage\\
%Translation & mRNA cleavage & 21 & mRNA cleavage\\
%Translation & mRNA  & 21--22 & mRNA cleavage\\
%Translation & mRNA  & 24--26 & Histone and DNA Modification\\
%\noalign{\smallskip}\hline\noalign{\smallskip}
%\end{tabular}
%$^a$ Table foot note (with superscript)
%\end{table}
%%
%\section{Section Heading}
%\label{sec:3}
%% Always give a unique label
%% and use \ref{<label>} for cross-references
%% and \cite{<label>} for bibliographic references
%% use \sectionmark{}
%% to alter or adjust the section heading in the running head
%Instead of simply listing headings of different levels we recommend to
%let every heading be followed by at least a short passage of text.
%Further on please use the \LaTeX\ automatism for all your
%cross-references and citations as has already been described in
%Sect.~\ref{sec:2}.

%Please note that the first line of text that follows a heading is not indented, whereas the first lines of all subsequent paragraphs are.

%If you want to list definitions or the like we recommend to use the Springer-enhanced \verb|description| environment -- it will automatically render Springer's preferred layout.

%\begin{description}[Type 1]
%\item[Type 1]{That addresses central themes pertainng to migration, health, and disease. In Sect.~\ref{sec:1}, Wilson discusses the role of human migration in infectious disease distributions and patterns.}
%\item[Type 2]{That addresses central themes pertainng to migration, health, and disease. In Sect.~\ref{subsec:2}, Wilson discusses the role of human migration in infectious disease distributions and patterns.}
%\end{description}

%\subsection{Subsection Heading} %
%In order to avoid simply listing headings of different levels we recommend to let every heading be followed by at least a short passage of text. Use the \LaTeX\ automatism for all your cross-references and citations citations as has already been described in Sect.~\ref{sec:2}.

%Please note that the first line of text that follows a heading is not indented, whereas the first lines of all subsequent paragraphs are.

%\begin{svgraybox}
%If you want to emphasize complete paragraphs of texts we recommend to use the newly defined Springer class option \verb|graybox| and the newly defined environment \verb|svgraybox|. This will produce a 15 percent screened box 'behind' your text.

%If you want to emphasize complete paragraphs of texts we recommend to use the newly defined Springer class option and environment \verb|svgraybox|. This will produce a 15 percent screened box 'behind' your text.
%\end{svgraybox}


%\subsubsection{Subsubsection Heading}
%Instead of simply listing headings of different levels we recommend to
%let every heading be followed by at least a short passage of text.
%Further on please use the \LaTeX\ automatism for all your
%cross-references and citations as has already been described in
%Sect.~\ref{sec:2}.

%Please note that the first line of text that follows a heading is not indented, whereas the first lines of all subsequent paragraphs are.

%\begin{theorem}
%Theorem text goes here.
%\end{theorem}
%%
%% or
%%
%\begin{definition}
%Definition text goes here.
%\end{definition}

%\begin{proof}
%%\smartqed
%Proof text goes here.
%\qed
%\end{proof}

%\paragraph{Paragraph Heading} %
%Instead of simply listing headings of different levels we recommend to
%let every heading be followed by at least a short passage of text.
%Further on please use the \LaTeX\ automatism for all your
%cross-references and citations as has already been described in
%Sect.~\ref{sec:2}.

%Note that the first line of text that follows a heading is not indented, whereas the first lines of all subsequent paragraphs are.
%%
%% For built-in environments use
%%
%\begin{theorem}
%Theorem text goes here.
%\end{theorem}
%%
%\begin{definition}
%Definition text goes here.
%\end{definition}
%%
%\begin{proof}
%\smartqed
%Proof text goes here.
%\qed
%\end{proof}
%%
%\begin{acknowledgement}
%If you want to include acknowledgments of assistance and the like at the end of an individual chapter please use the \verb|acknowledgement| environment -- it will automatically render Springer's preferred layout.
%\end{acknowledgement}
%%
%\section*{Appendix}
%\addcontentsline{toc}{section}{Appendix}
%%
%%
%When placed at the end of a chapter or contribution (as opposed to at the end of the book), the numbering of tables, figures, and equations in the appendix section continues on from that in the main text. Hence please \textit{do not} use the \verb|appendix| command when writing an appendix at the end of your chapter or contribution. If there is only one the appendix is designated ``Appendix'', or ``Appendix 1'', or ``Appendix 2'', etc. if there is more than one.

%\begin{equation}
%a \times b = c
%\end{equation}

%%%%%%%%%%%%%%%%%%%%%%%%% referenc.tex %%%%%%%%%%%%%%%%%%%%%%%%%%%%%%
% sample references
% %
% Use this file as a template for your own input.
%
%%%%%%%%%%%%%%%%%%%%%%%% Springer-Verlag %%%%%%%%%%%%%%%%%%%%%%%%%%
%
% BibTeX users please use
 %\bibliographystyle{}
 %\bibliography{}
%
%\biblstarthook{References may be \textit{cited} in the text either by number (preferred) or by author/year.\footnote{Make sure that all references from the list are cited in the text. Those not cited should be moved to a separate \textit{Further Reading} section or chapter.} The reference list should ideally be \textit{sorted} in alphabetical order -- even if reference numbers are used for the their citation in the text. If there are several works by the same author, the following order should be used: 
%\begin{enumerate}
%\item all works by the author alone, ordered chronologically by year of publication
%\item all works by the author with a coauthor, ordered alphabetically by coauthor
%\item all works by the author with several coauthors, ordered chronologically by year of publication.
%\end{enumerate}
%The \textit{styling} of references\footnote{Always use the standard abbreviation of a journal's name according to the ISSN \textit{List of Title Word Abbreviations}, see \url{http://www.issn.org/en/node/344}} depends on the subject of your book:
%\begin{itemize}
%\item The \textit{two} recommended styles for references in books on \textit{mathematical, physical, statistical and computer sciences} are depicted in ~\cite{science-contrib, science-online, science-mono, science-journal, science-DOI} and ~\cite{phys-online, phys-mono, phys-journal, phys-DOI, phys-contrib}.
%\item Examples of the most commonly used reference style in books on \textit{Psychology, Social Sciences} are~\cite{psysoc-mono, psysoc-online,psysoc-journal, psysoc-contrib, psysoc-DOI}.
%\item Examples for references in books on \textit{Humanities, Linguistics, Philosophy} are~\cite{humlinphil-journal, humlinphil-contrib, humlinphil-mono, humlinphil-online, humlinphil-DOI}.
%\item Examples of the basic Springer style used in publications on a wide range of subjects such as \textit{Computer Science, Economics, Engineering, Geosciences, Life Sciences, Medicine, Biomedicine} are ~\cite{basic-contrib, basic-online, basic-journal, basic-DOI, basic-mono}. 
%\end{itemize}
%}

\begin{thebibliography}{99.}%


    \bibitem{bochereau1990}Bochereau, Laurent, and Bourgine, Paul. A Generalist-Specialist Paradigm for
Multilayer Neural Networks. Neural Networks, 1990.


\bibitem{binaryconnect}Courbariaux, Matthieu, Bengio, Yoshua, and David, Jean-Pierre. BinaryConnect:
Training Deep Neural Networks with Binary Weights during Propagations. NIPS,
2015.


\bibitem{galaxy}Dieleman, Sander, Willett, Kyle W., and Dambre, Joni. Rotation-invarient
convolutional neural networks for galaxy morphology prediction. Oxford Journals,
2015.


\bibitem{deepspeech2}Hannun, Awni, Case, Carl, Casper, Jared, Catanzaro, Bryan, Diamos, Greg, Elsen,
Erich, Prenger, Ryan, Satheesh, Sanjeev, Sengupta, Shubho, Coates, Adam, and Ng,
Andrew Y. Deep Speach: Scaling up end-to-end speech recognition. Arxiv Preprint,
2014.


\bibitem{darkknowledge}Hinton, Geoffrey E., Vinyals, Oriol, and Dean, Jeff. Distilling th Knowledge in
a Neural Network. NIPS 2014 Deep Learning Workshop.


\bibitem{emonets}Kahou, Samira Ebrahimi, Bouthiller, Xavier, Lamblin, Pascal, Gulcehre, Caglar,
Michalski, Vincent, Konda, Kishore, Jean, S�bastien, Froumenty, Pierre, Dauphin,
Yann, Boulanger-Lewandowski, Nicolas, Ferrari, Raul Chandias, Mirza, Mehdi,
Warde-Farley, David, Courville, Aaron, Vincent, Pascal, Memisevic, Roland, Pal,
Christopher, and Bengio, Yoshua. EmoNets: Multimodal deep learning approaches
for emation recofnition in video. Journal on Mutlimodal User Interfaces, 2015.


\bibitem{cifar}Krizhevsky, Alex. Learning Multiple Layers of Features from Tiny Images. 2009.


\bibitem{spectral}Ng, Andrew Y., Jordan, Micheal I., Weiss, Yair. On spectral clustering: Analysis
and an algorithm. NIPS 2002.


\bibitem{imagenet}Russakovsky, Olga, Deng, Jia, Su, Hao, Krause, Jonathan, Satheesh, Sanjeev, Ma,
Sean, huang, Zhiheng, Karpathy, Andrej, Khosla, Aditya, Bernstain, Michael,
Berg, Alexander C., and Fei-Fei, Li. ImageNet Large Scale Visual Recognition
Challenge. International Journal of Computer Vision, 2015.


\bibitem{vgg}Simonyan, Karen and Zisserman, Andrew. Very Deep Convolutional Networks for
Large-Scale Image Recognition. International Conference on Learning
Representations, 2015.


\bibitem{allcnn}Springenberg, Jost Tobias, Dosovitskiy, Alexey, Brox, Thomas, and Riedmiller,
Martin. Striving for Simplicity: The All Convolutional Net. International
Conference on Learning Representations Workshop, 2015.


\bibitem{seq2seq}Sutskever, Ilya, Vinyals, Oriol, and Le, Quoc V. Sequence to Sequence Learning with
Neural Networks. Arxiv Preprint, 2014.


\bibitem{inception}Szegedy, Christian, Liu, Wei, Jia, Yangqing, Sermanet, Pierre, Reed, Scott,
Anguelov, Dragomir, Erhan, Dumitru, Vanhoucke, Vincent, and Rabinovich, Andrew.
Going deeper with convolutions. Arxiv Preprint, 2014.


\bibitem{wardefarley}Warde-Farley, David, Rabinovich, Andrew, and  Anguelov, Dragomir. Self-Informed
Neural Networks Structure Learning. International Conference on Representations
Learning, 2015.

%@inproceedings{bochereau1990,
  %title={A generalist-specialist paradigm for multilayer neural networks},
  %author={Bochereau, Laurent and Bourgine, Paul},
  %booktitle={Neural Networks, 1990., 1990 IJCNN International Joint Conference on},
  %pages={87--91},
  %year={1990},
  %organization={IEEE}
%}

%@inproceedings{binaryconnect,
  %title={BinaryConnect: Training Deep Neural Networks with binary weights during propagations},
  %author={Courbariaux, Matthieu and Bengio, Yoshua and David, Jean-Pierre},
  %booktitle={Advances in Neural Information Processing Systems},
  %pages={3105--3113},
  %year={2015}
%}

%@article{galaxy,
  %title={Rotation-invariant convolutional neural networks for galaxy morphology prediction},
  %author={Dieleman, Sander and Willett, Kyle W and Dambre, Joni},
  %journal={Monthly Notices of the Royal Astronomical Society},
  %volume={450},
  %number={2},
  %pages={1441--1459},
  %year={2015},
  %publisher={Oxford University Press}
%}

%@article{deepspeech2,
  %title={Deep Speech 2: End-to-End Speech Recognition in English and Mandarin},
  %author={Amodei, Dario and Anubhai, Rishita and Battenberg, Eric and Case, Carl and Casper, Jared and Catanzaro, Bryan and Chen, Jingdong and Chrzanowski, Mike and Coates, Adam and Diamos, Greg and others},
  %journal={arXiv preprint arXiv:1512.02595},
  %year={2015}
%}

%@article{darkknowledge,
  %title={Distilling the knowledge in a neural network},
  %author={Hinton, Geoffrey and Vinyals, Oriol and Dean, Jeff},
  %journal={arXiv preprint arXiv:1503.02531},
  %year={2015}
%}

%@article{emonets,
  %title={Emonets: Multimodal deep learning approaches for emotion recognition in video},
  %author={Kahou, Samira Ebrahimi and Bouthillier, Xavier and Lamblin, Pascal and Gulcehre, Caglar and Michalski, Vincent and Konda, Kishore and Jean, S{\'e}bastien and Froumenty, Pierre and Dauphin, Yann and Boulanger-Lewandowski, Nicolas and others},
  %journal={Journal on Multimodal User Interfaces},
  %pages={1--13},
  %publisher={Springer}
%}

%@misc{cifar,
  %title={Learning multiple layers of features from tiny images},
  %author={Krizhevsky, Alex},
  %year={2009},
  %publisher={Citeseer}
%}

%@article{spectral,
  %title={On Spectral Clustering: Analysis and an algorithm},
  %author={Ng, Andrew Y and Jordan, Michael I and Weiss, Yair}
%}

%@article{imagenet,
  %title={Imagenet large scale visual recognition challenge},
  %author={Russakovsky, Olga and Deng, Jia and Su, Hao and Krause, Jonathan and Satheesh, Sanjeev and Ma, Sean and Huang, Zhiheng and Karpathy, Andrej and Khosla, Aditya and Bernstein, Michael and others},
  %journal={International Journal of Computer Vision},
  %volume={115},
  %number={3},
  %pages={211--252},
  %year={2015},
  %publisher={Springer}
%}

%@article{vgg,
  %title={Very deep convolutional networks for large-scale image recognition},
  %author={Simonyan, Karen and Zisserman, Andrew},
  %journal={arXiv preprint arXiv:1409.1556},
  %year={2014}
%}

%@article{allcnn,
  %title={Striving for simplicity: The all convolutional net},
  %author={Springenberg, Jost Tobias and Dosovitskiy, Alexey and Brox, Thomas and Riedmiller, Martin},
  %journal={arXiv preprint arXiv:1412.6806},
  %year={2014}
%}

%@inproceedings{seq2seq,
  %title={Sequence to sequence learning with neural networks},
  %author={Sutskever, Ilya and Vinyals, Oriol and Le, Quoc V},
  %booktitle={Advances in neural information processing systems},
  %pages={3104--3112},
  %year={2014}
%}

%@inproceedings{inception,
  %title={Going deeper with convolutions},
  %author={Szegedy, Christian and Liu, Wei and Jia, Yangqing and Sermanet, Pierre and Reed, Scott and Anguelov, Dragomir and Erhan, Dumitru and Vanhoucke, Vincent and Rabinovich, Andrew},
  %booktitle={Proceedings of the IEEE Conference on Computer Vision and Pattern Recognition},
  %pages={1--9},
  %year={2015}
%}

%@article{wardefarley,
  %title={Self-informed neural network structure learning},
  %author={Warde-Farley, David and Rabinovich, Andrew and Anguelov, Dragomir},
  %journal={arXiv preprint arXiv:1412.6563},
  %year={2014}
%}




% and use \bibitem to create references.
%
% Use the following syntax and markup for your references if 
% the subject of your book is from the field 
% "Mathematics, Physics, Statistics, Computer Science"
%
% Contribution 
%\bibitem{science-contrib} Broy, M.: Software engineering --- from auxiliary to key technologies. In: Broy, M., Dener, E. (eds.) Software Pioneers, pp. 10-13. Springer, Heidelberg (2002)
%%
%% Online Document
%\bibitem{science-online} Dod, J.: Effective substances. In: The Dictionary of Substances and Their Effects. Royal Society of Chemistry (1999) Available via DIALOG. \\
%\url{http://www.rsc.org/dose/title of subordinate document. Cited 15 Jan 1999}
%%
%% Monograph
%\bibitem{science-mono} Geddes, K.O., Czapor, S.R., Labahn, G.: Algorithms for Computer Algebra. Kluwer, Boston (1992) 
%%
%% Journal article
%\bibitem{science-journal} Hamburger, C.: Quasimonotonicity, regularity and duality for nonlinear systems of partial differential equations. Ann. Mat. Pura. Appl. \textbf{169}, 321--354 (1995)
%%
%% Journal article by DOI
%\bibitem{science-DOI} Slifka, M.K., Whitton, J.L.: Clinical implications of dysregulated cytokine production. J. Mol. Med. (2000) doi: 10.1007/s001090000086 
%%
%\bigskip

% Use the following (APS) syntax and markup for your references if 
% the subject of your book is from the field 
% "Mathematics, Physics, Statistics, Computer Science"
%
% Online Document
%\bibitem{phys-online} J. Dod, in \textit{The Dictionary of Substances and Their Effects}, Royal Society of Chemistry. (Available via DIALOG, 1999), 
%\url{http://www.rsc.org/dose/title of subordinate document. Cited 15 Jan 1999}
%%
%% Monograph
%\bibitem{phys-mono} H. Ibach, H. L\"uth, \textit{Solid-State Physics}, 2nd edn. (Springer, New York, 1996), pp. 45-56 
%%
%% Journal article
%\bibitem{phys-journal} S. Preuss, A. Demchuk Jr., M. Stuke, Appl. Phys. A \textbf{61}
%%
%% Journal article by DOI
%\bibitem{phys-DOI} M.K. Slifka, J.L. Whitton, J. Mol. Med., doi: 10.1007/s001090000086
%%
%% Contribution 
%\bibitem{phys-contrib} S.E. Smith, in \textit{Neuromuscular Junction}, ed. by E. Zaimis. Handbook of Experimental Pharmacology, vol 42 (Springer, Heidelberg, 1976), p. 593
%%
%\bigskip
%%
%% Use the following syntax and markup for your references if 
%% the subject of your book is from the field 
%% "Psychology, Social Sciences"
%%
%%
%% Monograph
%\bibitem{psysoc-mono} Calfee, R.~C., \& Valencia, R.~R. (1991). \textit{APA guide to preparing manuscripts for journal publication.} Washington, DC: American Psychological Association.
%%
%% Online Document
%\bibitem{psysoc-online} Dod, J. (1999). Effective substances. In: The dictionary of substances and their effects. Royal Society of Chemistry. Available via DIALOG. \\
%\url{http://www.rsc.org/dose/Effective substances.} Cited 15 Jan 1999.
%%
%% Journal article
%\bibitem{psysoc-journal} Harris, M., Karper, E., Stacks, G., Hoffman, D., DeNiro, R., Cruz, P., et al. (2001). Writing labs and the Hollywood connection. \textit{J Film} Writing, 44(3), 213--245.
%%
%% Contribution 
%\bibitem{psysoc-contrib} O'Neil, J.~M., \& Egan, J. (1992). Men's and women's gender role journeys: Metaphor for healing, transition, and transformation. In B.~R. Wainrig (Ed.), \textit{Gender issues across the life cycle} (pp. 107--123). New York: Springer.
%%
%% Journal article by DOI
%\bibitem{psysoc-DOI}Kreger, M., Brindis, C.D., Manuel, D.M., Sassoubre, L. (2007). Lessons learned in systems change initiatives: benchmarks and indicators. \textit{American Journal of Community Psychology}, doi: 10.1007/s10464-007-9108-14.
%%
%%
%% Use the following syntax and markup for your references if 
%% the subject of your book is from the field 
%% "Humanities, Linguistics, Philosophy"
%%
%\bigskip
%%
%% Journal article
%\bibitem{humlinphil-journal} Alber John, Daniel C. O'Connell, and Sabine Kowal. 2002. Personal perspective in TV interviews. \textit{Pragmatics} 12:257--271
%%
%% Contribution 
%\bibitem{humlinphil-contrib} Cameron, Deborah. 1997. Theoretical debates in feminist linguistics: Questions of sex and gender. In \textit{Gender and discourse}, ed. Ruth Wodak, 99--119. London: Sage Publications.
%%
%% Monograph
%\bibitem{humlinphil-mono} Cameron, Deborah. 1985. \textit{Feminism and linguistic theory.} New York: St. Martin's Press.
%%
%% Online Document
%\bibitem{humlinphil-online} Dod, Jake. 1999. Effective substances. In: The dictionary of substances and their effects. Royal Society of Chemistry. Available via DIALOG. \\
%http://www.rsc.org/dose/title of subordinate document. Cited 15 Jan 1999
%%
%% Journal article by DOI
%\bibitem{humlinphil-DOI} Suleiman, Camelia, Daniel C. O�Connell, and Sabine Kowal. 2002. `If you and I, if we, in this later day, lose that sacred fire...�': Perspective in political interviews. \textit{Journal of Psycholinguistic Research}. doi: 10.1023/A:1015592129296.
%%
%%
%%
%\bigskip
%%
%%
%% Use the following syntax and markup for your references if 
%% the subject of your book is from the field 
%% "Computer Science, Economics, Engineering, Geosciences, Life Sciences"
%%
%%
%% Contribution 
%\bibitem{basic-contrib} Brown B, Aaron M (2001) The politics of nature. In: Smith J (ed) The rise of modern genomics, 3rd edn. Wiley, New York 
%%
%% Online Document
%\bibitem{basic-online} Dod J (1999) Effective Substances. In: The dictionary of substances and their effects. Royal Society of Chemistry. Available via DIALOG. \\
%\url{http://www.rsc.org/dose/title of subordinate document. Cited 15 Jan 1999}
%%
%% Journal article by DOI
%\bibitem{basic-DOI} Slifka MK, Whitton JL (2000) Clinical implications of dysregulated cytokine production. J Mol Med, doi: 10.1007/s001090000086
%%
%% Journal article
%\bibitem{basic-journal} Smith J, Jones M Jr, Houghton L et al (1999) Future of health insurance. N Engl J Med 965:325--329
%%
%% Monograph
%\bibitem{basic-mono} South J, Blass B (2001) The future of modern genomics. Blackwell, London 
%
\end{thebibliography}

%\end{document}
